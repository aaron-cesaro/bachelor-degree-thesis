\documentclass[11pt]{thesistemp}
\usepackage[italian]{babel}
\usepackage[utf8]{inputenc}
\usepackage{hyperref}
\usepackage[export]{adjustbox}
\usepackage{subcaption}
\usepackage{graphicx}
  \graphicspath{ {../media/} }
\usepackage{wrapfig}
\usepackage{color}
  \definecolor{dkgreen}{rgb}{0,0.6,0}
  \definecolor{gray}{rgb}{0.5,0.5,0.5}
  \definecolor{mauve}{rgb}{0.58,0,0.82}
\usepackage{listings}
\lstset{
  frame=tb,
  aboveskip=3mm,
  belowskip=3mm,
  showstringspaces=false,
  columns=flexible,
  basicstyle={\tiny},
  numbers=none,
  numberstyle=\tiny\color{gray},
  keywordstyle=\color{blue},
  commentstyle=\color{dkgreen},
  stringstyle=\color{mauve},
  breaklines=true,
  breakatwhitespace=true,
  tabsize=3
}

\input{../resources/solidity-highlighting.tex}


%----------------------------------------------------------------------------------------
%	TITLE
%----------------------------------------------------------------------------------------

\title{Servizi di supporto alle transazioni su Blockchain Ethereum}
\creationdate{10 Novembre 2018}
\redactedby{Aaron Cesaro}
\begin{document}
\maketitle
\pagebreak
%----------------------------------------------------------------------------------------
%	ABSTRACT
%----------------------------------------------------------------------------------------
\begin{abstract}
Questo documento si presenta come Tesi di Laurea triennale per il corso di laurea in informatica dello studente Aaron Cesaro. In esso \`e contenuta una relazione dettagliata dell’attivit\`a di stage svolta presso l’azienda Sgame SA, situata a Lugano (Svizzera). Durante il tirocinio, della durata complessiva di 300 ore, ogni abiettivo prefissato \`e stato raggiunto con successo, permettendo allo studente di acquisire dettagliate conoscenze sulle tecnologie e le metodologie di sviluppo utilizzate dalla società.
Particolare enfasi è stata data alla comprensione ed all'utilizzo della blockchain \textit{Ethereum} e di tutti i sevizi di supporto necessari alla corretta integrazione dello stesso all'interno della piattaforma \textit{Sgame Pro}.
\end{abstract}

\tableofcontents

%----------------------------------------------------------------------------------------
%	INTRODUZIONE
%----------------------------------------------------------------------------------------

\section{Introduzione}

\subsection{Contenuto del documento}

All'interno del docuemento verrano trattati i seguenti argomenti: 
\begin{itemize}
	\item \textbf{Capitolo 2: Blockchain network:} verrà descritta la rete \textit{blockchain}, le tecnologie che ne permettono il funzionamento, le logiche di base su cui si fondano le transazioni e le principali differenze tra le piattaforme \textit{Bitcoin} ed \textit{Ethereum}.
	\item \textbf{Capitolo 3: La piattaforma Sgame Pro:} in questo capitolo verrà presentata l'applicazione \textit{Sgame Pro} e l'utilizzo di \textit{Ethereum} all'interno della stessa. Verrà inoltre data una visione d'insieme sulla piattaforma oltre a giustificare le motivazioni alla base dello sviluppo del servizio \textit{Token Value}.
	\item \textbf{Capitolo 4: Token Value service:}
	\item \textbf{Capitolo 5: Considerazioni finali:}
\end{itemize}


\subsection{Norme tipografiche}


%----------------------------------------------------------------------------------------
%	BLOCKCHAIN
%----------------------------------------------------------------------------------------
\pagebreak
\section{Blockchain network}

\subsection{Cos'è una Blockchain}

\subsubsection{Definizione}

\textit{Blockchain} è una tecnologia che permette la creazione ed amministrazione di un grande database
distribuito tramite la gestione di transazioni condivisibili tra più nodi di una rete peer-to-peer.
Si tratta quindi di un database strutturato in blocchi (\textit{block}) che sono tra loro collegati (\textit{chain}) in modo che ogni transazione avviata sulla rete debba essere validata dalla rete stessa. 
In estrema sintesi la \textit{blockchain} è rappresentata da una catena di blocchi che
contengono e gestiscono più transazioni facendo uso della crittografia per rendere sicuro
l’immagazzinamento di dati ed il trasferimento di strumenti di valuta.\\\\
Pur essendo quest'ultima la definizione ``formale'' di \textit{blockchain}, ritengo che essa non evidenzi con chiarezza e semplicità cosa effettivamente una \textit{blockchain} sia.\\
Proverò quindi a scomporre ed analizzare la prima parte della definizione per rendere più fruibile il concetto.

\subsubsection{Server based e P2P}
La rete che normalmente viene utilizzata tutti i giorni per navigare in internet è quasi sempre \textit{server based} [\textbf{Figura~\ref{fig:server-based-net}}].\\
La peculiarità di questo sistema è che tutte le informazioni sono contenute in un solo posto, il \textit{Server} (da qui il nome \textit{server based}), il quale spesso gestisce un database che esercita il ruolo di archivio per l'immagazzinamento di dati ed effettua su di essi ricerche qualora vengano richiesti.
%----------------------------------------------------------------------------------------
%	FIGURA 1
%----------------------------------------------------------------------------------------
\begin{figure}[h]
    \centering
    \begin{subfigure}[h]{0.3\textwidth}
        \includegraphics[width=\textwidth]{server-based-net.png}
        \caption{Server based network}
        \label{fig:server-based-net}
    \end{subfigure}\qquad
    \begin{subfigure}[h]{0.3\textwidth}
        \includegraphics[width=\textwidth]{p2p-based-net.png}
        \caption{P2P based network}
        \label{fig:p2p-based-net}
    \end{subfigure}
    \caption{Tipologie di network}
    \label{fig:net}
\end{figure}\\\\
A differenza di una rete \textit{server based}, in una rete \textit{peer-to-peer} [\textbf{Figura~\ref{fig:p2p-based-net}}] (anche detta \textit{P2P}) non esiste la presenza di un server centrale che invia informazioni e tutti i dati vengono scambiati direttamente tra i nodi collegati alla rete. Ogni utente è quindi un \textit{client} ed un \textit{server} contemporaneamente.
Proprio per questa duplice funzione ogni dispositivo connesso viene detto \textit{nodo} della rete.\\
La sostanziale differenza tra le due tipologie di rete risiede nel fatto che mentre nel primo caso il \textit{server} contiene tutte le informazioni, nel secondo sono tutti e soli i \textit{client} a contenere i dati.\\
Ciò significa che nel primo caso il proprietario del \textit{server} può aggiungere, modificare o eliminare i dati che sono contenuti in esso, mentre nel secondo, anche se un nodo cancella o modifica i propri dati, gli altri nodi conterranno comunque tutte le informazioni originali. Da qui il termine distribuito.\\
\`E ora possible riprendere in mano la definizione originale: \textit{blockchain} è una tecnologia che permette di creare e gestire un grande archivio di informazioni, non contenute in un unico posto, ma del quale esiste una copia in ogni nodo connesso alla rete.

\subsection{Blockchain e Bitcoin}

\subsubsection{Ledger}
In letteratura è ritenuto più intuitivo usare \textit{Bitcoin} (\textit{BTC}) per esporre, tramite esempi semplificati, il funzionamento di una \textit{blockchain}.\\
Un \textit{Bitcoin} è una singola unità di valuta digitale che, proprio come l'\textit{Euro} non ha valore intrinseco, se non quello intenzionalmente attribuitogli grazie al consenso di scambio per l'acquisizione di beni o servizi.\\
%----------------------------------------------------------------------------------------
%	FIGURA 2
%----------------------------------------------------------------------------------------
\begin{wrapfigure}{r}{0.35\textwidth}
	\vspace{-31pt}
	\begin{center}
    	\includegraphics[width=0.37\textwidth]{ledger.png}
  	\end{center}
  	\vspace{-10pt}
  	\caption{Ledger}
  	\label{fig:ledger}
  	\vspace{-30pt}
\end{wrapfigure}
Per tenere traccia della quantità di \textit{Bitcoin} che ogni utente possiede si utilizza ciò che viene definito un \textit{Ledger} (libro mastro), che altro non è che un file in cui viene tenuta traccia di tutte le transazioni.
Il \textit{ledger} non è contenuto in un server centrale come ad esempio quello di una banca, ma ne esiste una copia in ogni nodo partecipante alla rete. In [\textbf{Figura~\ref{fig:ledger}}] è riportato, anche se in modo estremamente semplificato, un esempio di \textit{ledger}. 
Anche se nella realtà un ledger è molto diverso da quello in figura, nella pratica il disegno rappresenta fedelmente la funzione principale ricoperta da ogni copia del \textit{ledger} posseduta dai singoli nodi.
Nella colonna \textbf{Account} viene riportato il nome del proprietario dei \textit{Bitcoin}, mentre nella colonna \textbf{Value} è indicata la quantità posseduta da ognuno dei partecipanti.\\\\\\\\\\\\
Mettiamo caso che David voglia inviare cinque \textit{Bitcoin} a Sandra.
Per farlo è necessario che David mandi un messaggio sulla rete, il quale contiene la richiesta di transazione ed il numero di Bitcoin da lui posseduti.
Come informazione aggiuntiva viene inoltre trasmessa la quantità di \textit{Bitcoin} che possederà Sandra nel caso in cui la transazione avesse luogo. 
Il messaggio viene raggiunto dai nodi vicini a David i quali aggiornano i propri ledger con il risultato della possibile transazione (cioè David -5 \textit{BTC} e Sandra +5 \textit{BTC}) e rinviano il messaggio ai nodi a loro adiacenti. 
In questo modo il messaggio si espande per tutta la rete.
%----------------------------------------------------------------------------------------
%	FIGURA 3
%----------------------------------------------------------------------------------------
\begin{figure}[h]\hfill
    \centering
    \includegraphics[width=\textwidth]{transaction.png}
    \caption{ Richiesta di transazione tra due nodi}
    \label{fig:transaction}
\end{figure}\\
Il fatto che il ledger sia mantenuto da tutti i nodi implica tre cose fondamentali, che stanno alla base del concetto della blockchain:
\begin{itemize}
\item  tutti sono a conoscenza di tutte le transazioni che avvengono sulla rete;
\item se la transazione non và a buon fine nessuno se ne prende la responsabilità in quanto non esiste un’entità centrale che si prenda carico dell’esito delle transazioni;
\item non esiste il bisogno di garanzie o fiducia in quanto la sicurezza è ottenuta tramite particolari funzioni matematiche estremamente sicure.
\end{itemize} 
\pagebreak

\subsubsection{Transazioni}
Perchè una transazione possa avere luogo è necessario ciò che viene definito un \textit{Wallet} (portafogli), ossia un software che permetta di depositare e scambiare scriptovaluta, tra cui \textit{Bitcoin}.
Poichè deve essere possibile solo ed esclusivamente al proprietario di un determinato \textit{Wallet} inviare i propri \textit{Bitcoin}, ogni \textit{Wallet} è protetto tramite una tecnica crittografica che usa una coppia di chiavi tra loro connesse.
Esse prendono il nome di chiave privata (\textit{private key}) e chiave pubblica (\textit{public key}).
%----------------------------------------------------------------------------------------
%	FIGURA 4
%----------------------------------------------------------------------------------------
\begin{figure}[h]\hfill
    \centering
    \includegraphics[width=\textwidth]{public-private-keys.png}
    \caption{Verifica della transazione tramite chiavi}
    \label{fig:pkey}
\end{figure}\\
Ogni messaggio in uscita da un singolo indirizzo viene criptato con una chiave privata il quale, una volta derivata la corrispondente chiave pubblica con cui è possibile identificare univocamente l'indirizzo di partenza (\textit{from}),deve poi essere validato dal nodo locale.
La validazione è necessaria per accertarsi che la transazione sia stata realmente messa sulla rete dal proprietario dell'account.
A questo punto solo i possessori della chiave pubblica associata potranno decifrare il messaggio.\\
Quando David vuole mandare 5 \textit{Bitcoin} a Sandra, deve inviare sulla rete il messaggio criptato con la sua chiave privata in modo che venga identificato come il possessore di un certo numero di \textit{Bitcoin} e sia di conseguenza l’unico a poter sbloccare il proprio \textit{Wallet}.
Tutti gli altri nodi validano la transazione, verificando tramite la chiave pubblica di David che la richiesta di inviare valuta sia effettivamente partita da lui.
In questo modo si ottiene la validazione della transazione.
In altre parole per poter inviare un \textit{Bitcoin} è necessario provare alla rete di essere i possessori dell'indirizzo da cui partono i \textit{Bitcoin}.\\
Nella rete inoltre non viene tenuto conto del bilancio dei singoli utenti, ma vengono semplecemente registrate le transazioni che avvengono.
La verifica di una transazione in questo modo si riduce semplicemente al controllo di tutte le transazioni passate, effettuate dall’utente che vuole inviare una certa somma di \textit{Bitcoin}.

\subsubsection{Blocchi}
Su una \textit{blockchain} le transazioni vengono ordinate tramite accorpamento con altre transazioni avvenute in un lasso di tempo definito.
In altre parole più transazioni vengono raggruppate insieme ed inserite dentro a quello che viene chiamato \textit{Block} (Blocco). 
Ogni \textit{block} contiene quindi un definito numero di transazioni ed un collegamento al nodo precedente. 
In questo modo si viene a creare una catena di blocchi molto simile ad una \textit{linked list}.
Da qui il nome \textit{blockchain} (catena di blocchi).
%----------------------------------------------------------------------------------------
%	FIGURA 5
%----------------------------------------------------------------------------------------
\begin{figure}[h]\hfill
    \centering
    \includegraphics[width=\textwidth]{blocks.png}
    \caption{Rappresentazione di una blockchain}
    \label{fig:blocks}
\end{figure}\\
Le transazioni contenute nello stesso blocco sono considerate come avvenute nello stesso lasso temporale, mentre le transazioni che ancora non sono state raggruppate in un blocco sono considerate \textit{Unconfirmed}, cioè non ancora validate.\\ 
Ogni nodo della rete può raggruppare più transazioni e creare un blocco, suggerendo alla rete di inserirlo come prossimo blocco della catena. \\
Per essere effettivamente inserito nella rete un blocco deve contenere la soluzione ad un complesso problema matematico che, per essere risolto richiede una grossa potenza di calcolo, ed un pò di fortuna.
La risposta altro non è che un numero e l'unico modo per sapere quale sia il numero corretto da inserire consiste nel provarli tutti.
Il nodo che per primo risolve il problema acquisisce il diritto di inserire il prossimo blocco sulla catena e lo invia a tutti i nodi adiacenti.

\subsubsection{Mining}
A questo punto sorge spontaneo una domanda, ossia: "Se i Bitcoin posseduti da un account sono il risultato della somma di tutte le transazioni inviate e ricevute da quell'account, come è possibile ottenere altri \textit{Bitcoin}?"\\
La risposta a questa domanda é: "tramite il \textit{mining}".\\
Il \textit{mining} è l'attività svolta dai nodi definiti \textit{miners}, i quali, tramite la risoluzione di un complesso problema matematico, validano i blocchi, permettendo così a tutti i partecipanti alla rete di inviare e ricevere transazioni.\\
La validazione di un blocco nella pratica è una attività molto dispendiosa, sia in termini di energia elettrica che di consumo di banda.\\
Perchè la catena possa proseguire (cioè perchè possano essere effettuate nuove transazioni) è necessario che i blocchi siano inseriti nella catena e per farlo è necessario risolvere questo problema matematico.\\
Il modo escogitato per ripagare chi indovina il numero che valida il blocco (cioè svolge il lavoro di \textit{miner}) è una ricompensa in \textit{Bitcoin} da parte della rete. 
Questa ricompensa è ciò che incentiva le persone a provvedere al necessario lavoro computazionale per far continuare la catena e mantenere la rete utilizzabile. Senza i \textit{miners} i blocchi non potrebbero essere validati, la catena si fermerebbe e le transazioni non potrebbero più avere luogo.

\pagebreak
\subsection{Ethereum}

\subsubsection{Cos'é Ethereum}

Come \textit{Bitcoin}, \textit{Ethereum} è una \textit{public blockchain}.\\
Sebbene ci siano alcune significative differenze tecniche tra le due, la distinzione più importante da notare è che \textit{Bitcoin} ed \textit{Ethereum} differiscono sostanzialmente per scopo e capacità.\\
Il \textit{Bitcoin} è stato lanciato come valuta alternativa, o moneta digitale, ed offre una particolare applicazione della tecnologia \textit{blockchain}, ossia un sistema di pagamento elettronico.\\
\textit{Ethereum} invece viene principalmnte utilizzato per applicazioni decentralizzate tramite l'utilizzo degli \textit{Smart Contract}.\\
A differenza di \textit{Bitcoin}, \textit{Ethereum} utilizza due concetti di \textit{token}: il primo prende il nome di \textit{Ether} e corrisponde alla "moneta" effettivamente scambiata tra gli utenti della rete, il secondo viene invece utilizzato per pagare i \textit{miners}, i quali includono le transazioni nei blocchi, e prende il nome di \textit{gas}.

\subsubsection{Smart Contract}

Uno \textit{Smart Contract} è un programma che contiene un insieme di regole a cui le parti interessate accettano di aderire.\\
Nel caso in cui le regole definite all'interno di uno smart contract siano soddisfatte l'accordo tra le parti viene automaticamente applicato.\\
Il codice di uno \textit{Smart Contract} facilita, verifica e impone la negoziazione o l'esecuzione di un accordo o di una transazione e corrisponde alla forma più semplice di automazione decentralizzata.\\
Il successo di \textit{Ethereum} (e la più grande differenza con \textit{Bitcoin}) dipende proprio dal concetto di \textit{Smart Contract}. grazie a cui è possible programmare una serie definita di azioni che vengono attuate se e solo se le condizioni in esso contenute vengono soddisfatte.

%----------------------------------------------------------------------------------------
%	Sgame Pro
%----------------------------------------------------------------------------------------

\section{La piattaforma Sgame Pro}
%----------------------------------------------------------------------------------------
%	FIGURA 5
%----------------------------------------------------------------------------------------
\begin{figure}[h]\hfill
    \centering
    \includegraphics[width=\textwidth]{sgamepro-logo.png}
    \label{fig:sgamepro}
\end{figure}

\subsection{Overview}

\textit{Sgame Pro} è un aggregatore di \textit{mobile games} di proprietà di \textit{Sgame SA}, con sede a Lugano, Svizzera.\\
In sviluppo dal 2016, Sgame Pro ha lanciato con successo la sua \textit{Alpha version} nel 2017, raggiungendo oltre 50.000 download senza alcuna spesa di marketing.\\
\textit{Sgame Pro} è interamente focalizzato sull'industria dei \textit{mobile games} e ha sviluppato due importanti innovazioni tecniche:
\begin{itemize}
	\item Consentire ai giocatori di essere remunerati con un nuovo utility token (\textit{SGM}) semplicemente giocando con il proprio cellulare ai titoli proposti;
	\item Aggregare il frammentato settore dei \textit{publisher}, indipendenti e non, in un'unica piattaforma di gioco \textit{one-stop-shop}.
\end{itemize}
 L'\textit{SGM} (token di tipo \textit{utility} basato sullo standard \textit{Ethereum ERC-20}) sarà l'unico modo per accedere ai servizi e beneficiare di tutte le funzinalità messe a disposizione degli utenti, oltre ad essere il metodo di pagamento unico per tutte le transazioni all'interno dell'ecosistema della piattaforma.\\
All'interno dell'applicazione i giocatori non solo avranno l'opportunità di trovare tutti gli ultimi titoli mobile resi disponibli, ma avranno anche l'opportunità di sfidare gli altri utenti sia in privato che tramite la funzionalità di \textit{Public Challenge}.\\
Quest'ultima innovazione è davvero dirompente dato che il 78\% del mercato dei giochi mobile è single player, mentre la maggior parte delle entrate derivano da giochi multiplayer.\\
Gli \textit{Influencers} di \textit{Sgame Pro} includono \textit{Pewdiepie}, \textit{Tweakbox} e molti altri, con oltre 80 milioni di seguaci altamente coinvolti e sparsi in tutto il mondo.\\ 

\subsection{Funzionalità principali}

\subsubsection{Challenges}

Le \textit{Challenges} sono una delle caratteristiche principali di \textit{Sgame Pro}.\\
Esse consentono infatti di sfidare gli altri giocatori all'interno della piattaforma e di guadagnare gli \textit{SGM} messi in palio per la sfida.\\
Le \textit{challenges} permettono inoltre di trasformare qualunque gioco da semplice \textit{single play} a \textit{multiplayer}, dando agli utenti la possibilità di sfidarsi in modalità asincrona.\\

%----------------------------------------------------------------------------------------
%	FIGURA 6
%----------------------------------------------------------------------------------------
\begin{figure}[h]\hfill
    \centering
    \includegraphics[width=\textwidth]{challenges.png}
        \caption{Funzionlità Challenges}
    \label{fig:challenges}
\end{figure}
All'interno dell'applicazione esistono due diverse tipologie di sfide:
\begin{itemize}
	\item \textbf{Private Challenges}: le quali consentono agli utenti di sfidarsi tra loro, scegliendo tutti i partecipanti alla competizione direttamente tra i propri \textit{Friends}, e mettendo in palio l'importo in \textit{SGM} desiderato;
	\item \textbf{Public Challenges}: le quali invece danno la possibilità di creare una sfida \textit{ad-hoc} a cui chiuque può partecipare, specificando il numero minimo/massimo di utenti necessario affinchè la sfida possa avere luogo.
\end{itemize}
\pagebreak

\subsubsection{Leaderboard}

La \textit{Leaderboard} è la classifica globale di punteggi ottenuti dagli utenti in uno specifico titolo.\\
Questa funzionalità consente ai giocatori di confrontare le proprie prestazioni con quelle degli avversari.\\
L'innovativo approccio di Sgame Pro a questa "classica" funzionalità consiste nell'esporre:
\begin{itemize}
	\item \textbf{Classifiche comuni tra iOS ed Android}: i giocatori possono finalmente competere tra loro indipendentemente dal sistema operativo utilizzato;
	\item \textbf{Classifiche premiate}: i giocatori che battono particolari record od obiettivi vengono premiati in \textit{SGM}.
\end{itemize}

\subsubsection{User Profile}
%----------------------------------------------------------------------------------------
%	FIGURA 7
%----------------------------------------------------------------------------------------
\begin{wrapfigure}{r}{0.2\textwidth}
	\vspace{-31pt}
	\begin{center}
    	\includegraphics[width=0.2\textwidth]{user-profile.png}
  	\end{center}
  	\vspace{-10pt}
  	\caption{Profilo}
  	\label{fig:user-profile}
  	\vspace{-60pt}
\end{wrapfigure}
Ogni giocatore o \textit{influencer} avrà il proprio profilo utente dedicato contenente una panoramica di punteggi, tempi di gioco ed altre statistiche utili.\\
Il profilo utente contiene inoltre le seguenti informazioni:
\begin{itemize}
	\item livello dell'utente;
	\item numero di amici e \textit{followers};
	\item statistiche delle \textit{challenges};
	\item titoli giocati di recente;
	\item dati sugli avversari;
	\item dattagli sui guadagni.
\end{itemize}
\pagebreak

\subsubsection{Marketplace}

La sezione \textit{Marketplace} permette agli utenti della piattaforma \textit{Sgame Pro} di acquistare \textit{digital goods} (coupons, free subscriptions etc..) direttamente all'interno dell'applicazione.\\
\`E inoltre possibile fissare dei "\textit{target}", ossia scegliere un particolare oggetto all'interno del \textit{marketplace} e porsi come obiettivo il raggiungimento della cifra necessaria al suo acquisto.

\subsubsection{Wallet}

All'interno della funzionalità \textit{Wallet} l'utente può sempre tenere sotto controllo il suo patrimonio ed avere una panoramica dettagliata del flusso di cassa (in \textit{SGM}) del suo account.\\
\\
Da questa sezione è inoltre possibile trasformare gli \textit{SGM} guadagnati all'interno della piattaforma in criptovaluta fruibile all'interno della rete \textit{Ethereum}.\\
\`E infatti tramite questa funzinalità che entrano in gioco la \textit{blockchain} e tutti i servizi di supporto ad essa associati.
\pagebreak

\subsection{Sgame Pro ed Ethereum}

L'idea iniziale alla base dell'integrazione di \textit{Ethereum} all'interno della piattaforma \textit{Sgame Pro} prevedeva un'applicazione completamente decentralizzata (\textit{full decentralized}) a cui ogni movimento di \textit{SGM} corrispondeva una transazione su \textit{Ethereum}.\\\\
Questo approccio avrebbe portato numerosi benefici a livello di \textit{trust}, ma sarebbe stato altamente insostenibile per il \textit{business model} della società.\\
Infatti, come già detto, ogni transazione su \textit{Ethereum} richiede una determinata quantità di \textit{gas}, necessaria per l'esecuzione dello \textit{smart contract}, che deve essere aggiunta all'ammontare che si desidera trasferire.
In altre parole il costo di ogni transazione sarebbe stato maggiore dell'ammontare scambiato durante la transazione stessa.\\
Si è deciso quindi di utilizzare un sistema di remunerazione interno "classico", ossia tramite assegnazione a livello di database, dando però la possibilità agli utenti di transferire i propri guadagni sul proprio \textit{wallet} personale qualora lo desiderassero.

\subsubsection{SGaMe Token}

\textit{Ethereum} è stato creato per essere un vero e proprio ambiente di sviluppo, infatti esso può essere utilizzato, tramite la creazione di appositi \textit{Smart Contract}, per creare criptovalute che si appoggino sulla sua \textit{currency} base, ossia l'\textit{Ether}.\\
Le criptovalute derivate da \textit{Ether}, dette anche \textit{token}, sono principalmente state utilizzate per l'attuazione delle \textit{ICO} (\textit{Initial Coin Offer}), ossia \textit{crowdfundig} attuati tramite la vendita di particolari criptovalute, utilizzate come finanziamenti al posto della moneta tradizionale (\textit{FIAT}).\\
Come già citato l'\textit{SGM} è il token utilizzato dalla piattaforma \textit{Sgame Pro}.\\
Esso è stato creato allo scopo di uniformare quelli che vengono definiti \textit{In-App purchase}, permettendo così agli utenti di poter acquistare qualunque tipo di beneficio, all'interno dei titoli presenti nella piattaforma, con una unica valuta.\\\\
L'\textit{SGM} si appoggia su quello che formalmente \textit{Ethereum} definisce lo \textit{Standard ERC-20}, ossia un'interfaccia che gli \textit{Smart Contract} possono implementare nel caso in cui si intenda utilizzare alcune tipologie di servizi offerti da \textit{Ethereum.}
%----------------------------------------------------------------------------------------
%	ERC-20 STANDARD
%----------------------------------------------------------------------------------------
\begin{lstlisting}[language=Solidity]
contract ERC20Interface {
   function totalSupply() public view returns (uint);
   function balanceOf(address tokenOwner) public view returns (uint balance);
   function allowance(address tokenOwner, address spender) public view returns (uint remaining);
   function transfer(address to, uint tokens) public returns (bool success);
   function approve(address spender, uint tokens) public returns (bool success);
   function transferFrom(address from, address to, uint tokens) public returns (bool success);

   event Transfer(address indexed from, address indexed to, uint tokens);
   event Approval(address indexed tokenOwner, address indexed spender, uint tokens);
}
\end{lstlisting}
L'interfaccia \textit{ERC-20} funge da contratto per il set minimo di funzionalità che si intende mettere a disposizione dopo la creazione (\textit{issuing}) ti un token derivato da \textit{Ethereum}.\\
La scelta di aderire a questo standard è stata guidata dall'altissima compatibilità con i vari \textit{Exchange} presenti oggigiorno sul mercato.

\section{Token Value Service}

\subsection{Overview}

\textit{Token Value} è un servizio di acquisizione e memorizzazione dello storico di valori di criptovalute sulle diverse piattaforme di scambio (\textit{Exchange}).\\
Il servizio espone delle \textit{API} pubbliche, utilizzate direttamente dalla piattaforma \textit{Sgame Pro}, per il calcolo delle vincite dei giocatori ed il prezzo dei beni acquistabili nella sezione \textit{Marketplace}. \\
Grazie all'implementazione di questo servizio il sistema può applicare le logiche di business basate sul calcolo del prezzo presentato all'interno dell'applicazione, in modo da assicurare equità nell'attribuzione di \textit{SGM} e coerenza del valore riportato sugli oggetti acquistabili all'interno della stessa.\\
Le funzionalità principali del servizio comprendono:
\begin{itemize}
	\item ottenimento del valore del token \textit{SGM} tramite l'utilizzo delle API messe a disposizione di ciascun exchange;
	\item memorizzazione persistente dei valori ottenuti su database PostgreSQL;
	\item caching dei valori per la diminuzione dei tempi di risposta;
	\item utilizzo e trasmissione di valori ottimali secondo le logiche di selezione approvate dalla società.
\end{itemize}
Il servizio rappresenta attualmente un tassello fondamentale per la società \textit{Sgame SA} in quanto il valore del token riportato e propagato all'intera applicazione ha un impatto estremamente concreto sui guadagni dell'intera società.\\\\
Una logica scorretta all'interno del servizio porterebbe irreparabilemente ad una erronea attribuizione delle vincite agli utenti e ad una possible perdita di denaro per la società.\\\\
Proprio per queste ragioni il servizio utilizza logiche estremamente semplici e facilmente testabili.
\pagebreak

\subsection{Pianificazione}

\begin{itemize}
	\item \textbf{Prima Settimana}(42,5 ore) – Incontro con il team di sviluppo, formazione sul funzionamento di Ethereum ed introduzione ai linguaggi C\# e Solidity;
	\item \textbf{Seconda Settimana}(42,5 ore) – Acquisizione delle informazioni necessarie alla realizzazione del servizio TokenValue ed ai servizi interni che utilizzerà ;
	\item \textbf{Terza Settimana}(42,5 ore) – Sviluppo della logica di acquisizione e memorizzazione persistente dei valori necessari all’utilizzo del servizio TokenValue.
	\item \textbf{Quarta Settimana}(42,5 ore) – Debugging e sviluppo della documentazione del servizio TokenValue. Entro il termine della settimana il tirocinante dovrà aver terminato e documentato il servizio sviluppato.
	\item \textbf{Quinta Settimana}(42,5 ore) – Creazione dell’interfaccia grafica necessaria agli analisti di Sgame per la visualizzazione dei valori di mercato della criptovaluta target.
	\item \textbf{Sesta Settimana}(42,5 ore) – Finalizzazione, bug fixing e documentazione dell’interfaccia grafica di TokenValue;
	\item \textbf{Settima Settimana}(42,5 ore) – Creazione dell’ambiente di testing su AWS EC2 tramite l’utilizzo del tool Ganache ed automazione del deploy dello Smart Contract utilizzato dalla piattaforma Sgame; Utilizzo della rete pubblica Ropsten per l'advanced testing dello Smart Contract necessario al funzionamento della piattaforma Sgame;
	\item \textbf{Ottava Settimana}(2,5 ore) – Presentazione agli Stakeholders del lavoro svolto sul servizio TokenValue, comprensivo di documentazione e giustificazione delle soluzioni adottate dal tirocinante.
\end{itemize}
\pagebreak

\subsection{Analisi dei requisiti}

\subsubsection{Notazioni}

Viene di seguito riportata la notazione che verrà utilizzata per l’identificazione dei requisiti classificati per utilità strategica: 
\begin{itemize}
	\item \textbf{Ob} – requisiti obbligatori, irrinunciabili per qualsiasi Stakeholders.
	\item \textbf{De} – requisiti desiderabili, aggiungono valore al prodotto.
\end{itemize}
Viene di seguito riportata la notazione che verrà utilizzata per l’identificazione dei requisiti classificati per verificabilità: 
\begin{itemize}
	\item \textbf{Vi} – requisiti vincolo, imposti dal cliente o dal sistema in cui lavora il software.
\end{itemize}

\subsubsection{Specifica dei requisiti}

\textbf{Obbligatori}:
\begin{itemize}
	\item \textbf{Ob001} – il prodotto \textit{Token Value} deve raccogliere i dati sui valori delle criptovalute tramite chiamata \textit{REST API} al servizio cryptocompare.
	\item \textbf{Ob002} – il prodotto \textit{Token Value} deve salvare dati persistenti sia sul valore attuale del token che sullo storico dei valori richiesti.
	\item \textbf{Ob003} – il prodotto \textit{Token Value} deve essere corredato da documentazione.
	\item \textbf{Ob004} – l’ambiente testnet di tipo private deve utilizzare uno script di \textit{auto-deploy} per lo \textit{Smart Contract}.
\end{itemize}
\textbf{Desiderabili}:
\begin{itemize}
	\item \textbf{De001} – il prodotto \textit{Token Value} deve essere documentato tramite\textit{ XML notation}.
	\item \textbf{De002} – il prodotto \textit{Token Value} deve essere facilmente configurabile sia in ambiente \textit{Development} che in ambiente \textit{Staging} tramite l’utilizzo di specifiche variabili d’ambiente.
\end{itemize}
\textbf{Vincolo}:
\begin{itemize}
	\item \textbf{Vi001} – il prodotto \textit{Token Value} deve essere realizzato nel linguaggio \textit{C\#} ed utilizzare \textit{PostgreSQL} per il salvataggio persistente dei dati.
	\item \textbf{Vi002} – l’ambiente \textit{testnet} di tipo \textit{private} deve utilizzare il tool \textit{Ganache}.
	\item \textbf{Vi003} – l’ambiente \textit{testnet} di tipo \textit{public} deve utilizzare il la rete \textit{Ropsten}.
\end{itemize}
\pagebreak

\subsection{Architettura}

L'architettura del servizio riprende fedelmente il modello utilizzato dagli altri servizi relativi ad \textit{Ethereum} sviluppati sulla piattaforma.\\\\
Ogni servizio è composto da due moduli separati:
\begin{itemize}
	\item \textbf{Public API}: interfaccia pubblica che espone \textit{API} di tipo \textit{REST} all'interno dell'ambiente stesso;
	\item \textbf{Core}: \textit{business logic} del servizio. Si occupa delle richieste verso i servizi esterni e dell'interfacciamento con il database. Nel modulo sono quindi compresi sia il \textit{Business Layer} che il \textit{Data Layer}.
\end{itemize}
%----------------------------------------------------------------------------------------
%	FIGURA 8
%----------------------------------------------------------------------------------------
\begin{figure}[h]\hfill
    \centering
    \includegraphics[width=\textwidth]{tkvalue-arch.png}
        \caption{Architettura del servizio Token Value}
    \label{fig:tkvalue-arch}
\end{figure}
Grazie a questa struttura a \textit{layers} si garantisce l'impossibilità di comunicazione diretta con la logica del servizio ed una netta separazione delle responsabilità all'interno del servizio stesso.\\
Il servizio \textit{Token Value} si interfaccia poi con il resto del backend dell'applicazione (quello che la compagnia ha deciso di chiamare \textit{Mobile API}), e con \textit{TX}, ossia il servizio che si occupa della comunicazione diretta con lo \textit{Smart Contract} interfacciandosi alla \textit{blockchain Ethereum}.

\subsection{Progettazione}
Viene di seguito esposto il diagramma delle classi (\textit{UML 2.0}) del servizio \textit{Token value}.\\
All'interno nel servizio è possibile mappare i due moduli riportati nello schema dell'architettura nel seguente modo:
\begin{itemize}
	\item \textbf{Public API} contiene la classe \textit{ValueController}, la quale funge da interfaccia interna e viene richiamata da \textit{Mobile API};
	\item \textbf{Core} comprende tutto il resto del servizio, ossia i \textit{Managers} ed i \textit{Repository}, necessari all'orchestrazione ed alla comunicazione con il database rispettivamente.
\end{itemize}
%----------------------------------------------------------------------------------------
%	FIGURA 9
%----------------------------------------------------------------------------------------
\begin{figure}[h]\hfill
    \centering
    \includegraphics[width=\textwidth]{TkUMLClassDiagram.png}
        \caption{Diagramma delle classi del servizio Token Value}
    \label{fig:TkUMLClassDiagram}
\end{figure}

\subsubsection{Tecnologie}

Per il rispetto del requisito [\textbf{Vi001}] la scelta del linguaggio di programmazione da utilizzare è inevitabilemente ricaduta su \textit{C\#}.\\
Questo requisito è dovuto al fatto che l'intera piattaforma è stata sviluppata con la tecnologia \textit{.NET Framework} di \textit{Microsoft} e poggia sul \textit{Cloud Azure}, anch'esso sviluppato e mantenuto dalla stessa \textit{Microsoft}.\\
Questa scelta ha reso possible una perfetta integrazione del servizio con tutte le estensioni (\textit{Insight, Metrics etc.. }) e funzionalità aggiuntive messe a disposizione dalla piattaforma.\\\\
Lo stesso requisito ha inoltre imposto la creazione del database tramite l'utilizzo di \textit{PostgreSQL}. Questa è stata presa semplicemente per coerenza tecnologica con gli altri servizi che già utilizzavano la stessa tipologia di database.

\pagebreak
\subsection{Sviluppo}
Per comprendere a pieno l'andamento dello sviluppo del servizio è necessario innanzitutto dare una panoramica sia delle metodologie e dei \textit{tools} utilizzati dalla società che del \textit{Flow di sviluppo} imposto a tutti gli sviluppatori.

\subsubsection{Metodologia di sviluppo}
Data l'elevata preparazione tecnica e professionalità degli sviluppatori presenti nel \textit{team} di \textit{Sgame Pro} e vista la necessità di continui miglioramenti dell'applicazione la società ha deciso di utilizzare una metodologia di sviluppo \textit{Agile} e, più in particolare, \textit{Scrum}.

\subsubsubsection{Scrum}
\subsubsection{Development flow}
La società \textit{Sgame SA} ha richiesto che il \textit{flow} di lavoro, durante l'implementazione di nuove \textit{features} di uno stesso servizio debba sempre utilizzare la seguente \textit{pipeline}:
\begin{enumerate}
	\item \textbf{Development};
	\item \textbf{UAT};
	\item \textbf{Staging};
	\item \textbf{Production}.
\end{enumerate}
La struttura sopra esposta segue fedelmente la struttura dei principali \textit{branch} riportata sull'apposito \textit{repository} presente in \textit{GitHub}.\\
Ad ogni punto sopra elencato corrispondono ambienti di sviluppo differenti, ognuno con uno scopo ben definito.\\\\
L'ambiente di \textit{Development} è dedicato esclusivamente all'implementazione di servizi, funzionalità e correzione degli errori (\textit{bug fix}).\\\\
\textit{UAT}, acronimo per \textit{User Acceptance Testing}, è utilizzato per il testing interno delle funzionalità e per l'approvazione interna del \textit{Product Manager}.\\\\
L'ambiente di \textit{Staging} è dedicato invece al \textit{Beta Testing}, ossia a test effettuati da personale esterno alla compagnia. Questo risulta estremamente efficace nello sviluppo di applicazioni mobile in quanto l'elevato numero di dispositivi presenti nel mercato richiede accertamenti (soprattutto lato \textit{UI}) sulla corretta visualizzazione e funazionamento delle nuove funzionalità.\\\\
\textit{Production} è invece l'ambiente "esterno", ossia dove l'applicazione viene utilizzata dagli utenti di tutto il mondo.


\subsubsubsection{Processo di implementazione delle features}
Una volta iniziato il lavoro su una nuova \textit{feature} del servizio è necessario innanzitutto creare un nuovo \textit{branch} il cui nome deve tassativamente seguire la seguente struttura:\\\\
\textit{Iniziali\_dello\_sviluppatore + \_nome\_feature}\\\\
Tutti i \textit{commit} riguardanti la stessa \textit{feature} andranno sempre ad utilizzare questo nuovo branch.\\
Una volta terminata, testata e validata la funzionalità o l'\textit{improvement} è necessario fare lo \textit{stash} di tutti i commit verso un unico commit che prenderà (come \textit{commit message}) il nome della \textit{feature} stessa.\\
Questa andrà quindi inserita sul \textit{branch} di \textit{development} dopo aver effettuato il \textit{rebase}.\\\\
Il procedimento è invece differente quando, una volta implementata e testata una nuova funzionalità sull'ambiente di \textit{development}, questa vada inserita sull'ambiente di \textit{UAT}, \textit{Staging} e \textit{Production}.\\
L'unico modo per modificare il codice presente in uno di questi tre ambienti dedicati è tramite \textit{pull request}, il quale titolo deve essere tassativamente il nome della \textit{feature} che si va ad inserire, ed si rivela assolutamente necessaria agli altri membri del team per effettuare \textit{Code review} sul codice che si intende inserire.\\
Questo permette un controllo efficace sul codice in ingresso nei vari ambienti tramite gli appositi sistemi di \textit{CI/CD}.


\pagebreak
\subsection{Integrazione}
%----------------------------------------------------------------------------------------
%	FIGURA 9
%----------------------------------------------------------------------------------------
\begin{figure}[h]\hfill
    \centering
    \includegraphics[width=\textwidth]{sgame-vpc.png}
        \caption{Architettura della piattaforma Sgame Pro}
    \label{fig:sgame-vpc}
\end{figure}


\bibliographystyle{abbrv}
\bibliography{thesis}

\end{document}
